
\begin{itemize}
  \item \cite{desai_aNovelAproachToCarryingOutMiniProjectInComputerScienceAndEngineering_2012} introduces 5'th semester mini-project with the aim to address
    \begin{itemize}
      \item majority of cap-stone projects ended up with poor designs and no testing (all focus on churning out code)
      \item high-failure rate and
      \item difficulty in guiding many different cap-stone projects with focus on how it is done, not what is done, significantly improved student's understanding, project management (e.g.\ the number of capstone projects which were completed in time),  and the quality of the capstone projects. Also found that ability to work in teams is improved.
    \end{itemize}
    Introducing 5'th semester mini-project
  \item \cite{isaacson_aMiniSoftwareEngineeringProjectForCs0_2003} introduce a software engineering mini project as part of a first year breadth-first introduction to computer science. Gave benefits to teams who completed task successfully and where all team members contributed in some major documented way. used time sheets. Teams of 25 including team manager, team architect, syadmin and team librarian (maintain and integrate code). Customer available to answer questions.
  \item Tadayon \cite{tadayon_softwareEngineeringBasedOnTheTeamSoftwareProcessWithARealWorldProject_2004} discusses using a semi-realistic industrial SE project to teach SE. XP with peer programming.
  \item Basholli et al.\ \cite{basholli_fairAssessmentInSoftwareEngineeringCapstoneProjects_2013} discusses peer assessment within group SE projects together with validation through questionaires as well as student feedback on the assessment. Clark et al.\ \cite{clark_selfAndPeerAssessmentInSoftwareEngineeringProjects_2005} also discuss assessment through peer reviews and self-assessment, individual contribution reports, and quantitative contribution assessment of group members by group members. Hayes et al.\ \cite{hayes_evaluatingIndividualContributionTowardGroupSoftwareEngineeringProjects_2003} validate./aigment this with pop questionaires for individuals about the project details.
  \item Collofello and Hart \cite{collofello_monitoringTeamProgressInASoftwareEngineeringProjectClass_1999} discuss monitoring team progress and individual contributions through regular weekly progress reports, team meeting reports, frequent deliverables, and metrics.
  \item Yu and Zhang \cite{yu_failureCaseStudy_anInstructuiveMethodForTeachingComputerNetworkEngineering_2010} show that using a failure case study is an effective way of teaching for a computer networking course (need better references for teaching through failure).
  \item Madsen and Desai \cite{madsen_failingToLearn_2010} argue organizations learn more effectively from failue than from success
  \item Varol and Bayrak \cite{varol_appliedSoftwareEngineeringEducation_2005} discuss the necessity of includinf a real-life client-sponsored project in a software engineering curriculum, particularly to covey the importance of SRS, analysis and design.
  \item Stein \cite{stein_usingLargeVsSmallGroupProjectsInCapstoneAndSoftwareEngineeringCourses_2002} discusses using large versus small groups in pedagogical SE projects
  \item van der Duim, Andersson and Sinnema \cite{vanderDuim_goodPracticesForEducationalSoftwareEngineeringProjects_2007} discuss some good practices for educational software engineering projects including the use of reciprocity (giving and takling) and cooperation amongst students, encourage contacts between students and faculty, providing prompt feedback, emphasizing time on task, active learning, communicating high expectations, and respecting diverse talents and different ways of learning (not a very useful reference, but might still be worth it)
  \item \cite{bolinger_fromStudentToTeacher_2010} discuss expanding software engineering project to include the wider business context and business case.
  \item \cite{peixoto_learningFromStudentsMistakesInSoftwareEngineeringCourses_2010} perform defect analysis $\rightarrow$ pattern identification $\rightarrow$  improvement definition as pedagogical process.
\end{itemize}