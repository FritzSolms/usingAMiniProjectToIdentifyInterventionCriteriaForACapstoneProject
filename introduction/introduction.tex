In order to prepare students for the work environment, it is customary for final year computer science  a and software engineering students to complete an industrial scale software development project \cite{oudshoorn_ExperienceWithAProjectBasedApproachToTeachingSoftwareEngineering_1994,roach_retrospectivesInASoftwareEngineeringProjectCourse_2011}. We have, however, found that students lack essential understanding of software engineering practices and tools as well as experience to work in teams leading to an excessively high failure rate \cite{}. 

In order to improve the pedagogical value of the final year software engineering module and reduce the probability of failure, we have inserted an intense six-week mini-project before the six-month capstone project\cite{vredasPapers,desai_aNovelAproachToCarryingOutMiniProjectInComputerScienceAndEngineering_2012}. The mini-project aims to introduce the use of configuration management and testing practices and tools, expose students to the complexities of team dynamics and the challenges around integrating independently developed modules into a single product. The latter is an area which is not typically included in the scope of a capstone project. To increase the experience students acquire in team dynamics, different teams with different challenges are formed for each phase of the project\cite{vredasPersonalityTraitsPaper}.

Within the mini-project, students are aimed to retrospectively analyze their strengths and weaknesses in both, technical domains and soft-skills\cite{roach_retrospectivesInASoftwareEngineeringProjectCourse_2011}. The latter is aimed to assist them when forming complementing teams for the capstone project. We have found that it is valuable to relax the aim of mini-project success in favour of optimizing the skills and insights acquired during the mini-project. In addition to assigning group marks for artifacts produced during the mini-project, students are also continuously assessed through peer reviews and self assessments \cite{clark_selfAndPeerAssessmentInSoftwareEngineeringProjects_2005}. Individual contribution marks are calculated from an assessment of the contributions to the version control system and from peer reviews \cite{hayes_evaluatingIndividualContributionTowardGroupSoftwareEngineeringProjects_2003}.

For the main project teams students form new teams of three students each. They submit tenders for projects with actual clients from either industry or academia, specifying the order of preference they have for those projects. Clients then select the teams they would prefer for their project in order of preference. Project allocations are based on both, team and client preferences. Teams are expected to follow a defined software development process of their choice in order to elicit the detailed requirements from the client and deliver the product to the client.

In the mean time the assessments (including peer reviews and self-assessments) are analyzed for interception indicators in order to proactively identify potential pathological teams which can be expected to experience difficulties during their project execution. In future these interception indicators will be used to proactively intervene in teams which are identified as potential pathological teams in order to make them aware of the challenges they might be facing and to assist them with introducing mitigating measures. This is expected to not only reduce the failure rate in capstone projects, but also to increase the pedagogical value of the course. This paper, however, focuses on identifying correlations between interception indicators deduced from mini-project observations and pathological aspects. 