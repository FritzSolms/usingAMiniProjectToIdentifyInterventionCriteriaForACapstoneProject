\begin{abstract}
In order to prepare computer science students for their six-month capstone industry project, we have introduced an intense six-week mini-project prior to the capstone project. The mini-project exposes students to many of the best practices and challenges for industrial software engineering projects. We have found that an analysis of the team dynamics, activities, peer reviews and self-assessments done during the mini-project can reveal information which can be used to identify intervention indicators for the capstone project, i.e.\  indicators which predict problems teams may be likely to experience during the execution of their capstone project. It is hoped that identifying intervention indicators may assist to trigger early intervention in order to improve the pedagogical value of the course and reduce the failure rate experienced in capstone projects. This paper reports on the identification of intervention criteria and the correlation of these criteria with pathological aspects observed in subsequent the capstone project.
\end{abstract}