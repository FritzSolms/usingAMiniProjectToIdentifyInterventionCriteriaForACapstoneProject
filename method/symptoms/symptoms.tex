In order to identify pathological project symptoms we used  various assessments, information obtained from the teaching assistant and client as well as self-assessments and peer reviews. The assessments included documentation assessments, code reviews, demo assessments, assessments made by the department's lecturers during the annual project day and client assessments. 

Very valuable information was obtained by integrating the teaching assistant into the various capstone teams. To this end, the teaching assistant requested to be invited to  with each team's chat group, took part in team meetings (e.g.\ scrum meetings), and had extensive discussions with individual team members, 

\todo[inline]{Add any other mentioning of your hard work, Stacey}.   

This enabled us to identify the following pathological symptoms in capstone projects
\begin{itemize}
  \item teams starting late with their capstone project (observed from the git repositories),
  \item team members working independently (observed by teaching assistant and peer reviews),
  
  \todo[inline]{Not sure about this, but if we do not currently ask how often they work together, we should do so next year.}, 
  
  \item certain team members slacking, i.e.\ contributing only very little to the project (observed from git repositories and peer reviews),
  \item certain members taking over key elements of the project to the exclusion of others,
  \todo[inline]{I am sure this can be formulated better.}
  \item teams do not follow a defined software development process (observed from documentation assessments and observations made by teaching assistant),
  \item low quality products (observed from demos, code reviews, continuous assessments, project day assessments and client feedback),
  \item weak quality assurance including weak or no unit testing or integration testing, no peer reviews, \ldots (observed from git repository reviews and discussions during demos),
  \item weak communication with the client (observed from peer reviews, client feedback and information obtained from group by teaching assistant),
  \item team tensions (observed from peer reviews and observations made by teaching assistant).
\end{itemize}


\todo[inline]{Do we want to state that these symptoms are simply based on past experience or what justification do we want to provide for these symptoms?}