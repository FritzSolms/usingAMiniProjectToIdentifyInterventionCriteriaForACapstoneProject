We identified a range of information sources from which we could extract information about the individual personalities and their performance during the mini-project. We then extracted measures from these sources which could potentially contribute to intervention indicators which represent some probability of the capstone project experiencing pathological symptoms.

\todo[inline]{Note that the intervention indicators are not the measures. For example, measuring 3 diligent isolates in a team could be an intervention indicator. We need to think about this \ldots}

Similarly we identified a set observed pathological symptoms of the capstone project. This information was also obtained from a variety of sources.

We then went ahead and performed a simple correlation analysis between each of the potential indicators and each of the pathological symptoms. 

\todo[inline]{Waffle a bit more \ldots}

We used a variety of sources to extract information from the mini-project which could potentially be used as intervention indicators for the capstone project. The sources include peer reviews, self-assessments, historical academic records, enrollment records, the version control system (git) and the teaching assistant who observed and assisted the teams. From this we were able to extract the following measures:
\begin{itemize}
  \item the academic strength (from the academic records),
  \item the work load the student had in the current academic year (from the enrollment records),
  \item the level of involvement (from the git repository analysis, the peer reviews and the self-assessments),
  \item the quality of work (from the git repository analysis, the peer reviews and the self-assessments),
  \item personality traits of the student (from peer reviews, self-assessments and observations made by the teaching assistant)
\end{itemize}

\todo[inline]{For each of the above measures we need to provide a little detail on how the corresponding value in the spreadsheet determined.}
